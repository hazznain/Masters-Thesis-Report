\chapter{Conclusion}
\label{chapter:concluion}

Global energy needs are continuously growing. The conventional methods for producing more energy to meet the demand pose a great threat to environment. Among other solutions, Energy efficiency has become a major tool for minimizing the need of producing more energy to cater for the growing demand. Inherently, the cause of improving energy efficiency relies on understanding the usage patterns, identifying the problematic areas, establishing the good energy consumption practices and rectify the faults to reduce energy leakages. The advancement in sensors, ubiquitous computing  and communication technologies has provided basis for effectively collecting the usage data to understand energy usage. The collected data needs to be processed to generate generates leads for improving energy efficiency. The quality of insights generated from data improves if we consider the current data in context to historic data. This means that data volume for processing will keep on increasing. There can be multiple sources of data so the data formats can also vary. Then on the use case basis, data processing requires flexibility for customization and variation in speed of data processing. These all data features refers to application of Big Data technologies for energy efficiency. 

Distributed parallel computing programming models like MapReduce provide the basic environment for handling Big Data. We leveraged on power of MapReduce using Apache Hadoop ecosystem tools to present an end to end Big Data analytics tool. Hadoop supports scalability to meet large volumes of data sets while the there are other tools that can integrate with Hadoop to process complexity in data. We used the model platform to process real life energy data and generated insights that can be used to improve the energy efficiency. The proposed model provides a plug and play environment for many other analytic tools to integrate on need basis. It is based entirely on open source software components and can be deployed using general purpose hardware or any cloud based model.

We observed the strong sensitivity in consumption of specific energy types to the ecological factors like external temperature.The visualization of the average daily consumption of each building suggested the respective use of the buildings. It also provided the information about base loads of the building. Optimizing the base load of the building improves the energy efficiency. We also calculated the energy efficiency of the buildings in our sample data set and classified them on the basis of calculated energy efficiency. The classification provided us a reference point to identify the buildings where we can focus to locate the possible faults, energy leakages and bad consumption practices. This classification can be an important tool for organization working on improving the energy efficiency as it can help to isolate the problematic buildings. Our results also show a dynamic behaviour of buildings energy efficiency performance during different months in the year. This particular insight can be use as another reference to isolate the causes for energy inefficiency for a target building.

We also compared different prediction models to forecast the future energy usage of different home appliances on basis of their previous usage. The prediction model it self can be useful for energy service providers to understand and plan for user specific demand. However, this can also be used as the first step towards building a decision support system for user to effectively use home appliances. The decision support system can predict the usage pattern and then recommend the best options on basis of configurable parameters e.g. best tariffs or best time to use green energy etc. 

In a nutshell, our research provides a proof of concept of how emerging Big Data technologies can be applied in energy industry to improve energy efficiency and data driven decision making. We presented and demonstrated the concept of a Big Data analytics platform an applied it to solve real life use cases from energy industry. The output of our research is a working Big Data analytics platform and the results generated advance analytics techniques use specifically for solving energy efficiency problems.  
     

