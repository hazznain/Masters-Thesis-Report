\pagenumbering{arabic} 
\setcounter{page}{1}
\chapter{Introduction}
\label{chapter:intro}

 In modern era, we have seen phenomenal increase in human dependency on information and communication technology (ICT). ICT-enabled products and services hasve transformed the way of life on the planet. We need and depend on ICT to fulfil our needs from basic physiological level to the human desire of being effective part of society. There are many research areas and opportunities that are emerging as bi-products of this continuous transformation. One of them is the availability of digital traces of human activities. Every time we use these services, we produce digital traces that can be recorded and analysed. Big Data refers to these digital traces of human activity. Ubiquity of computing resources, fast and highly mobile connectivity and advent of social media usage has caused a great surge in volumes of data. Realizing the true potentials of data, businesses are not only utilizing it as source of decision making, but as a new revenue stream. Emerging large scale opportunities are reshaping the business models of many companies around the globe.
 
To support this transfiguration, we have seen a rapid development in distributed parallel computing, data communication software and machine learning. Industry giants such as Google and Yahoo have opened technologies and tools e.g. MapReduce and Hadoop to facilitate these advancements. Open source software communities like Apache Software foundation have further developed these tools to provide a complete ecosystem for handling Big Data and generate useful insights. The new specialized big data companies such as Cloudera and Hortonworks have emerged as catalyst for this data revolution. In this research, we try to formulate a model for end-to-end Big Data analytics platform based on these technologies that can ingest data from heterogeneous sources, process it in an efficient way, mine the data to generate the insights based on business logic and then present the information using interactive visualizations. This practical part of the research includes the development as well as implementation of the mentioned Big Data platform to perform analysis on real life use cases and generate useful insights. The model is based on open source software components available free of charge. There are other closed source software alternatives that can fit into the presented model, but the discussions about these solutions are not included in this document.

The topic of research is inspired by European Union's Cities as drivers of social change (CIVIS) project under 7th framework. CIVIS project focuses on adoption of ICT tools and techniques for integrating social aspects of city life into production, distribution and consumption of energy. It aims to make city life as a functional unit for improving energy efficiency. The use of pervasive ubiquitous computing is driving the smart energy solutions. The smart energy devices as part of this ecosystem generate high volumes of data. This data needs to be instantaneously transferred, stored, analysed and visualized for knowledge discovery and  improvements of services. The platform that was developed as part of this endeavour has the capability to automate the whole process.
 
The data from smart energy devices was analysed to detect the usage patterns and classify buildings on the basis of energy efficiency. Evaluation of some prediction models for energy consumption of household appliances was also included in the scope of research . These use cases provide the basis for designing, planning and implementing schemes for improving energy related services for  achieving higher efficiency in both production and usage. The insights generated from these use cases can also help in educating the consumer about benefits of energy efficiency and spread the awareness about behavioural changes from which the society and the individuals can benefit. 

This research is also supported by Technical Research Centre of Finland (VTT) as part of their Green Campus initiative. This project focuses on use of ICT based solutions for management and control systems to optimize the energy consumption without compromising the indoor environment of the buildings. VTT is also a supporter and partner of CIVIS project. VTT has installed specialized smart devices in selected test sites. VTT has contributed to our research by providing the data generated by these smart devices. VTT has also helped in scoping the use cases for energy efficiency with the experience and the knowledge they have gained from the related projects and research.

In a nutshell, this thesis focuses on providing a solution for collecting, storing, analysing and visualizing data generated by smart energy device for generating insights about energy consumption patterns and discovering the performance of different building units in terms of energy efficiency. The data analysis part of our research provides the models for knowledge discovery that can be used to improve energy efficiency at both producers and consumers ends. The big data analytics platform developed as part of this project is not limited to be used only for energy efficiency. It has the capability of handling other big data uses cases. However, within scope of this document we shall only discuss its use for energy usage patterns detection and efficiency.     





\section{Problem statement}

Energy efficiency can help to curtail production of energy to meet growth in demand. This in turn can help to reduce CO\(_{2}\) emissions. To achieve this goal we need to understand and improve the energy efficiency on both producer and consumer ends. ICT enabled smart energy grids and devices are being installed globally to measure energy consumption and improve energy efficiency. These smart devices produce large volumes of data. The data generated by different devices is in different formats. For purpose of knowledge discovery, this data needs to be collected, stored and analysed. The extracted insights from the analysis needs to be visualized for easy and effective understanding . The challenge gets even tougher when data needs to be collected and analysed in real time. Then with the time, volume of data and scope of analysis is expected to increase. In order to respond to the above mentioned challenges, a highly scalable and flexible data analysis platform is required that can automate the whole process. This platform needs to be very cost effective for global adaptation.
 
In scope of this research we provide a model for big data analytics platform that can provide the solution for these requirements. We also implement the proposed model and test it with real life data from smart energy devices. The proposed solution is based on open source components that can be deployed on general purpose hardware that can be procured very easily and inexpensively. The proposed platform can be scaled according to data requirements and additional functional components can be integrated as per the scope of analysis. The data analysis within our research also provides advance analytics models to extract the information based on energy efficiency use cases from large volumes of data.


\section{Helpful hints}

For referencing, we use Vancouver system \cite{neville2012referencing}. For discussing from authors point of view we use Author(s) name, year of publication along with reference numbers of bibliography. In case of quotation from author we use double quotes e.g. ``quotation from author''.

Throughout the document we discuss about energy. Due to our main focus, the term energy in our research and within this document refers to electricity or electric power. In case of all other types of energy we specifically mention the type name along with energy as a generic term.

In this document we discuss about the concept, development and use of a big data platform as our main environment for data analysis within our research. The terms like platform, data platform, and Big Data platform refers to the same concept. In case of a specific need of any other platform, we provide proper descriptions.

The mathematical text and the code snippets are differentiated from rest of the text using variation in font such as \(for\; mathematical\;text\) \& \begin{lstlisting}
code snippets font. \end{lstlisting}


\section{Structure of the thesis document}
\label{section:structure} 
This document has three parts i.e. header, body and tail. The header contains a tittle page, a cover page with an abstract of the content in the body part, an acknowledgement page followed by table of contents.

The main body of this document is divided into seven chapters. First chapter provides the introduction about the topic of research and the problem that we are trying to solve. Second chapter explains the main theoretical concepts and motivation for this research. We thoroughly discuss the other similar research endeavours in this chapter and their respective linkages to our research. The third and fourth chapters describe our methodology and practical implementation steps respectively. Fifth chapter provides the details of the data analysis and results that we produced. Chapter six presents a critical analysis of our approach and  results. It also provides some possible directions for further research on this topic. The last chapter provides a summary of what we have achieved from our efforts and how can it be used in real.

The tail part contains bibliography with the list of sequenced references and appendices. Appendices contain the additional supplementary information to support content in the body of the document.  


