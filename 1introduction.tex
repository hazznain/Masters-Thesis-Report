\chapter{Introduction}
\label{chapter:intro}

 In modern era we have seen phenomenal increase in human dependency on information and communication technology (ICT) enabled products and services. It has transformed the way of life on the planet. From fulfilling very basic physiological needs like health and food to the human needs of communicating with others and being part of wider social groups, we need and depend on ICT. There are many research areas and opportunities that are emerging as bi-products of this continuous transformation. One of them is the availability of digital traces of human activities. With every instance of use of these services we produce a digital trace that can be recorded and analysed. Big Data is a term that is being widely used to refer to these digital traces of human activity. Ubiquity of computing resources, fast and highly mobile connectivity and advent of social media usage has caused a great surge in volumes of data. Realizing the true potentials of data, businesses are not only utilizing it as source of decision making but new revenue lines and opportunities are emerging that are reshaping the business models of many companies around the globe.
 
To support this transfiguration, we have seen a rapid development in distributed parallel computing, data communication software and machine learning. Industry giants like Google and Yahoo has opened technologies and tools like MapReduce and Hadoop to facilitate these advancement and open source software communities like Apache Software foundation has further developed the tools to provide a complete ecosystem for handling big data and generate insights. The new specialized big data companies like Cloudera and Hortonworks has emerged that has acted as catalyst for this data revolution. In this research we try to formulate a model for end to end big data analytics platform based on these technologies that can ingest data from heterogeneous sources, process it in an efficient way, mine the data to generate the insights based on business logic and then present the information using interactive visualizations. This thesis includes the development as well as implementation of the mentioned big data platform to perform analysis on real life use case and generate useful insights. The model that we present in this thesis is based on open source software components available free of charge. There are other closed source software alternatives that can fit into the presented model but they are not discussed in this scope of this thesis.

This thesis is also part of European Union CIVIS- Cities as drivers of social change project under 7th framework. CIVIS project focuses on adaption of ICT tools and techniques for low carbon smart energy grid, distributed energy and information flow. The use of pervasive ubiquitous computing is driving the smart energy solutions. Combined with internet of things (IoT) for home/building automation, smart commuting, and remote monitoring is becoming the basis for energy conservation and energy efficiency. All the smart energy devices as part of this ecosystems generates high volumes of data, that needs to be instantaneously transferred, stored, analysed and visualized for knowledge discovery and  improvements of services for the goal of achieving high energy efficiency. The platform that was developed as part of this thesis has the capability to automate the whole process.
 
Energy usage pattern detection, classification of buildings on basis of energy efficiency and a prediction model for energy consumption per household will be the use cases for validating the developed big data analytics platform. These use cases also provide the basis for designing, planning and implementing schemes for improving energy related services for sake of achieving higher efficiency in both production and usage that contributes to cause of green environment in terms of less C02 emissions.  The insight generated from these use cases can also help in educating the consumer about benefits of energy conservation and spread the awareness about behavioural changes that can benefit society as well as individuals themselves. 

This master thesis is also supported by VTT, Technical Research Centre of Finland as part of their Green Campus initiative that focuses on use of ICT based solutions for innovative energy management and control systems capable to optimize the consumption without compromising the indoor environment. VTT is also a supporter and partner of CIVIS project. VTT has installed specialized smart devices in selected test sites that are the buildings owned by Aalto University. VTT has contributed to this thesis by providing the data generated by these smart devices. VTT has also helped in scoping for the use cases for energy efficiency by the experience and the knowledge they have from the related projects and research.

In a nutshell, this thesis focuses on providing a solution for collecting, storing, analysing and visualizing data generated by smart energy device for generating insights about energy consumption patterns and discovering the performance of different building units in terms of energy efficiency. This thesis also provides the models for knowledge discovery that can be used to improve energy efficiency at both producers and consumers ends. The big data analytics platform developed as part of this thesis is not limited to be used only for energy efficiency. It has the capability of handling other big data uses cases as well but we shall discuss its use for energy pattern detection and usage efficiency only in scope of this thesis report.     





\section{Problem statement}

Energy conservation is required to reduce C02 emissions from energy production and usage. To achieve this goal we need to understand and improve the energy efficiency on both producer and consumer end. ICT enabled smart energy grids and devices are being rolled out globally to measure energy consumption and improve on energy efficiency. These smart devices produce high volumes of data that may or may not be predicted and planned at time of setting up the infrastructure. The data generated by different devices comes in different formats. For knowledge discovery from this data it is required to collect, store analyse the data and then visualize the generated insights so the information can be understood efficiently. The challenge gets even tougher when data needs to be collected and analysed in real time. Then with the time, volume of data and scope of analysis is expected to increase. So to cater for all this a highly scalable and flexible data analysis platform is required that can automate the whole process. This platform needs to be very cost effective for global adaptation.
 
In scope of this research we provide a model for big data analytics platform that can provide the solution for these requirements. We also implement the proposed model and test it with real life energy smart devices data and use cases. The proposed solution is based on open source components that can be deployed on general purpose commercially available, hence it is very cost effective. The proposed platform can be scaled according to data volumes and additional functional components can be integrated as per the scope of analysis.


\section{Helpful hints}

Read the information from the university master's thesis
pages~ before starting the thesis.  You
should also go through the thesis grading
instructions~ together with your instructor and/or
supervisor in the beginning of your work.

\section{Structure of the Thesis}
\label{section:structure} 

You should use transition in your text, meaning that you should help
the reader follow the thesis outline. Here, you tell what will be in
each chapter of your thesis. 

